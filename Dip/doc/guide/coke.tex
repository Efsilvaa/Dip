\subsection{The Coke Supply Chain Problem (coke.py)} \label{sbs:coke}

This case study is sourced from the Operations Research Web in the Department of Engineering Science TWiki \cite{coke} (and was originally adapted from Leyland et al. \cite{geoff_coke}). There are 6 coal mines that produce coal. The coal is transported from the 6 mines to a coke-making plant where it is converted to coke using ``thermal decomposition''. Every tonne of coke produced by thermal decomposition requires 1.3 tonnes of coal. From the coke-making plants the coke is transported to one of 6 customers. There are 6 locations where coke-making plants can be constructed. There are 6 different size plants that can be constructed at each location.

The size of a plant determines the coke processing level in kilotonnes/year the plant can produce. \Tabref{tab:plant_size} shows the different plant sizes with their corresponding processing levels and construction cost in million RMB.
\begin{table}[htp]
\begin{center}
\begin{tabular}{|c|c|c|}
\hline
\multirow{2}{*}{\bf Plant Size} & {\bf Processing Level} & {\bf Cost} \\
 & {\bf (kT/year)} & {\bf (MRMB)} \\
\hline
1 & 75 & 4.4 \\
2 & 150 & 7.4 \\
3 & 225 & 10.5 \\
4 & 300 & 13.5 \\
5 & 375 & 16.5 \\
6 & 450 & 19.6 \\
\hline
\end{tabular}
\caption{Possible plant sizes} \label{tab:plant_size}
\end{center}
\end{table}

To get this problem into Dippy we use the PuLP modelling language. The entire input file is given below with a summary for each fragment.

\begin{enumerate}
\item Load PuLP and Dippy;
\lstinputlisting[firstnumber=5,linerange=5-6]{../../examples/coke.py}

\item Define the coke-from-coal conversion rate and a big-$M$ variable for use in later constraints;
\lstinputlisting[firstnumber=8,linerange=8-9]{C:/COIN/Dippy/examples/coke.py}

\item Define the supply of coal at the mines, the possible locations and construction costs of the coke-making plants and the demand for coke from the customers.
\lstinputlisting[firstnumber=11,linerange=11-22]{../../examples/coke.py}
\newpage
\lstinputlisting[firstnumber=24,linerange=24-45]{../../examples/coke.py}

\item Define the transportation costs from the mines to the coke-making plants and the coke-making plants to the customers in two tables (we also define a function {\tt read\_table} to read these tables, but this function is omitted for brevity);
\lstinputlisting[firstnumber=47,linerange=47-65]{../../examples/coke.py}

\item Read the data from the tables;
\lstinputlisting[firstnumber=79,linerange=79-81]{../../examples/coke.py}

\item Define the arcs and their costs from the mine $\rightarrow$ plant and plant $\rightarrow$ customer costs;
\lstinputlisting[firstnumber=83,linerange=83-86]{../../examples/coke.py}

\item \label{itm:2d} Define a 2-dimensional set for the plant sizes at each location;
\lstinputlisting[firstnumber=88,linerange=88-89]{../../examples/coke.py}

\item Create a \texttt{DipProblem} (extended from \texttt{LpProblem} in PuLP). Add binary variables that determine the plant sizes at each location and (non-negative) integer variables that determine the flow (in coal from the mines to the plants and coke from the plants to the customers) transported through the network;
\lstinputlisting[firstnumber=91,linerange=91-99]{../../examples/coke/coke.py}

\item Add the objective of minimising total cost = building costs (converted from MRMB to RMB) + transportation costs;
\lstinputlisting[firstnumber=101,linerange=101-106]{../../examples/coke.py}

\item Add constraints that limit the flow of coke out of a coke-making plant depending on the capacity plant constructed;
\lstinputlisting[firstnumber=108,linerange=108-113]{../../examples/coke.py}

\item Add constraints that limit the number of coke-making plants built at any single location to be one (Note. there is a size with capacity 0 if no plant will be built);
\lstinputlisting[firstnumber=115,linerange=115-118]{../../examples/coke.py}

\item Add constraints to conserve flow at the mines ($\leq$ supply), coke-making plants (flow in = coke-from-coal conversion rate $\times$ flow out) and customers ($\geq$ demand);
\lstinputlisting[firstnumber=120,linerange=120-133]{../../examples/coke.py}

\item Solve the \ac{MILP} problem using \ac{DIP} and display the solution in tabular form;
\lstinputlisting[firstnumber=160,linerange=160-178]{../../examples/coke.py}
\end{enumerate}

The preceding Python code defines and solves the Coke Supply Chain Problem. The solution takes 1.09s of CPU time and creates a tree using 201 nodes. The output defines plants to be built at locations 1, 5 and 6 and also defines shipments of coal and coke between the mines, plants and customers:
\begin{verbatim}
Build L1 150 (1.0)
Build L2 0 (1.0)
Build L3 0 (1.0)
Build L4 0 (1.0)
Build L5 450 (1.0)
Build L6 300 (1.0)

        L1       L2      L3      L4      L5      L6
M1      0.0      0.0     0.0     0.0     0.0     25.8
M2      0.0      0.0     0.0     0.0     0.0     340.475
M3      124.1175 0.0     0.0     0.0     585.0   0.0
M4      0.0      0.0     0.0     0.0     0.0     0.0
M5      0.0      0.0     0.0     0.0     0.0     0.0
M6      0.0      0.0     0.0     0.0     0.0     0.0

        C1       C2      C3      C4      C5      C6
L1      83.0     0.0     6.975   0.0     0.0     5.5
L2      0.0      0.0     0.0     0.0     0.0     0.0
L3      0.0      0.0     0.0     0.0     0.0     0.0
L4      0.0      0.0     0.0     0.0     0.0     0.0
L5      0.0      0.0     0.0     0.0     450.0   0.0
L6      0.0      5.5     0.0     5.5     270.75  0.0
\end{verbatim}
