%Using a high-level modelling language such as AMPL, GAMS, Xpress-Mp or OPL Studio enables Operations Research practitioners to express complicated \ac{MILP} problems relatively quickly and naturally compared to expressing them with a traditional computer programming language.
Using a high-level modelling language such as AMPL, GAMS, Xpress-MP or OPL Studio enables Operations Research practitioners to express complicated \ac{MILP} problems quickly and naturally.
Once defined in one of these high-level languages, the \ac{MILP} can be solved using one of a number of solvers.
However these solvers are not effective for all problem instances due to the computational difficulties associated with solving \ac{MILP}s (an NP-Hard class of problems).
Despite steadily increasing computing power and algorithmic improvements for the solution of \ac{MILP}s in general, in many cases problem-specific techniques need to be included in the solution process to solve problems of a useful size in any reasonable time.

Both commercial solvers -- such as CPLEX and Gurobi -- and open source solvers -- such as CBC, Symphony and \acs{DIP} (all from the COIN-OR repository \cite{coin_or}) -- provide callback functions that allow user-defined routines to be included in the solution framework.
To make use of these callback functions the user must first create their \ac{MILP} problem in a low-level computer programming language (C, C++ or Java for CPLEX, C, C++, C\#, Java or Python for Gurobi, C or C++ for CBC, Symphony or \acs{DIP}).
%While defining their problem they need to create structures to keep track of appropriate constraints and/or variables for later use in their user-defined routines.
As part of the problem definition, it is necessary to create structures to keep track of the constraints and\slash{}or variables.
Problem definition in C\slash{}C++\slash{}Java for a \ac{MILP} problem of any reasonable size and complexity is a major undertaking and thus a major barrier to the development of customised \ac{MILP} frameworks by both practitioners and researchers.

Given the difficulty in defining a \ac{MILP} problem in a low-level language, another alternative for problem formulation is to use a high-level mathematical modelling language.
By carefully constructing an indexing scheme, constraints and\slash{}or variables in the high-level language can be identified in the low-level callback functions.
However implementing the indexing scheme can be as difficult as using the low-level language to define the problem in the first place and does little to remove the barrier to solution development.

The purpose of the research presented here is to demonstrate a tool, Dippy, that supports easy experimentation with and customisation of advanced \ac{MILP} solution frameworks. 
To achieve this aim we needed to:
\begin{enumerate}
\item provide a modern high-level modelling system that enables users to quickly and easily describe their \ac{MILP} problems;
\item enable simple identification of constraints and variables in user-defined routines in the solution framework.
\end{enumerate}
The first requirement is satisfied by the modelling language PuLP~\cite{pulp}.
Dippy extends PuLP to use the \ac{DIP} solver, and enables user-defined routines, implemented using Python and PuLP, to be accessed by the \ac{DIP} callback functions. This approach enables constraints or variables defined in the \ac{MILP} model to be easily accessed using PuLP in the user-defined routines.
In addition to this, \ac{DIP} is implemented so that the \ac{MILP} problem is defined the same way whether branch-and-cut or branch-price-and-cut is being used -- it hides the implementation of the master problem and subproblems.
This makes it very easy to switch between the two approaches when experimenting with solution methods.
All this functionality combines to overcome the barrier described previously and provides researchers, practitioners and students with a simple and integrated way of describing problems and customising the solution framework.

The rest of this article is structured as follows.
In \scnref{scn:overview} we provide an overview of the interface between PuLP and \ac{DIP}, including a description of the callback functions available in Python from \ac{DIP}.
In \scnref{scn:techs} we describe how Dippy enables experimentation with improvements to \ac{DIP}'s \ac{MILP} solution framework by showing example code for a common problem.
We conclude in \scnref{scn:concl} where we discuss how this project enhances the ability of researchers to experiment with approaches for solving difficult \ac{MILP} problems. We also demonstrate that \ac{DIP} (via PuLP and Dippy) is competitive with leading commercial (Gurobi) and open source (CBC) solvers.
